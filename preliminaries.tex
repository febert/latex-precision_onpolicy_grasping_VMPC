\vspace{-0.1cm}
\section{Preliminaries}
\label{sec:prelim}
\vspace{-0.2cm}

Our visual MPC problem formulation follows the problem statement outlined in prior work~\cite{foresight}. In this setting, an action-conditioned video prediction model $g$, typically represented by a deep neural network, is used to predict future camera observations $\hat{I}_{1:T} \in \mathbb{R}^{T \times H\times W \times 3}$, conditioned on a sequence of candidate actions $a_{1:T}$, where the prediction horizon is $T$. This can be written as $\hat{I}_{1:T} = g(a_{1:T}, I_0)$, where $I_0$ is the frame from the current time-step. An optimization-based planner is  used to select the action sequence that results in an outcome that accomplishes a user-specified goal. This type of vision-based control is highly general, since it reasons over raw pixel observations without the requirement for a fixed-size state space, and has been demonstrated to generalize effectively to non-prehensile manipulation of previously unseen objects~\cite{foresight,sna}.

Visual MPC assumes the task can be defined in terms of pixel motion. In the initial image $I_0$ we define $n$ source pixel locations denoted by the coordinates $d_{0,i} \in \mathbb{N}^2$ (for $i \in [0,..n]$) and the analogous for the goal image $I_g$ denoted by $d_{g,i} \in \mathbb{N}^2$. Given a goal, visual MPC plans for a sequence of actions $a_{1:T}$
to move the pixel at $d_{0,i}$ to $d_{g,i}$. If this pixel lies on top of an object, this corresponds to moving that entire object to a goal position. Note that this problem formulation resembles visual servoing, but it is considerably more complex, since moving the object at $d_0$ might require complex non-prehensile or prehensile manipulation and long-horizon reasoning.
The planning problem is formulated as the minimization of a cost function $c$, which in accordance with prior work \cite{sna}, measures the distance between the predicted pixel positions $\hat{d}_{\tau}$ and the goal position $d_g$ for each pixel $i$:
\begin{align}
c = \sum^n_{i = 1}  \lambda_i c_i && c_i = \sum_{\tau = 1, \dots, T} \mathbb{E}_{\hat{d}_{\tau,i} \sim P_{\tau,i}} \left[\|\hat{d}_{\tau,i} - d_{g,i}\|_2\right]  
\label{eq:cost}
\end{align}
where $c_i\in \mathbb{R}$ are the costs per source pixel, $\lambda_i$ are weighting factors discussed in section \ref{sec:reg} and $P_{\tau,i}$ is the distribution over predicted pixel positions. The advantage of distance-based cost functions is that they are well-shaped and can be optimized efficiently. 

In this paper we use the video prediction model architecture developed by~\cite{savp}, where future images are generated by transforming past images. Starting with a distribution over initial positions of the designated pixel \mbox{$P_{t_0,i}\in\mathbb{R}^{H\times W}, \sum_{H,W} P_{t_0,i} = 1$} at time $t = 0$, the model predicts distributions over its positions $P_{t,i}$ at time $t \in \{ 1, \dots, T \}$ by exploiting the image transformations used to generate future frames. Planning is performed by sampling candidate actions sequences and optimizing using the cross-entropy method (CEM) \cite{cem-rk-13} to achieve the lowest possible cost $c$.

To obtain the best results with imperfect models, the action sequence is replanned at each real-world time step\footnote{We refer to timesteps in the real world as $t$ and to predicted time-steps as $\tau$.} $t \in \{0,...,t_{max}\}$ following the framework of model-predictive control (MPC): at each real-world step $t$, the first action of the best action sequence is executed. 
At the first real-world time step $t=0$, the distribution $P_{\tau=0,i}$ is initialized as 1 at the location of the designated pixel and zero elsewhere. In prior work \cite{sna, foresight}, in subsequent steps ($t > 0$),  the prediction of the previous step is used to initialize $P_{\tau=0,i}$. However this causes accumulating errors, often preventing the model from solving long-term tasks or responding to situations where the outcome of an action was different than expected. In effect, the model loses track of which object was designated in the initial image.






